%%%%%%%%%%%%%%%%%%%%%%%%%%%%%%%%%%%%%%%%%
% Poem
% LaTeX Template
% Version 1.0 (2/11/2015)
%
% This template has been downloaded from:
% http://www.LaTeXTemplates.com
%
% Original author:
% Vel (vel@latextemplates.com)
%
% License:
% CC BY-NC-SA 3.0 (http://creativecommons.org/licenses/by-nc-sa/3.0/)
%
% General notes:
% 1) All lines in a verse environment must end with \\, the last verse in a stanza
% must end in \\!
% 2) This template is based on the verse package, see the package documentation
% included with the template for further customisation options
% 
%%%%%%%%%%%%%%%%%%%%%%%%%%%%%%%%%%%%%%%%%

%----------------------------------------------------------------------------------------
%	DOCUMENT CONFIGURATIONS AND INFORMATION
%----------------------------------------------------------------------------------------

\documentclass[11pt, a4paper]{article} % Document font size and paper size

\usepackage{verse} % Required for typesetting poems - this package drives this template
\usepackage{geometry}
\usepackage[T1]{fontenc} % International character encodings
\usepackage{palatino} % Use the Palatino font by default
\usepackage[swedish]{babel}
\usepackage[utf8]{inputenc}

%\usepackage{stix} % Alternative Stix font

\setlength{\parindent}{0pt} % Disable paragraph indentation

% Author styles
\newcommand{\poemauthorcenter}[1]{\nopagebreak{\centering\footnotesize\textsc{#1}\par}} % Author as a footnote at the end of the poem center aligned
\newcommand{\poemauthorright}[1]{\nopagebreak{\raggedleft\footnotesize\textsc{#1}\par}} % Author as a footnote at the end of the poem aligned right

\renewcommand{\poemtitlefont}{\normalfont\bfseries\large\centering} % Define the poem title style

\setlength{\stanzaskip}{0.75\baselineskip} % The distance between stanzas

\pagestyle{empty} % Stop page numbering through the document

\begin{document}

%------------------------------------------------------------------------------
%	
%----------------------------------------------------------------------------------------


MTSen handlar inte bara om att ha kul och dricka vin, man måste äta också.
\begin{description}
  \item[Förrätt] The first item
  \item[Huvudrätt] The second item
  \item[Efterrätt] The third etc \ldots
\end{description}

\bigskip

och dricka
\begin{description}
  \item[Vin] The first item
  \item[Öl] The second item
  \item[Annat] The third etc \ldots
\end{description}




\newgeometry{left=1cm,right=1cm}
\twocolumn 

\poemtitle{Livet}

\settowidth{\versewidth}{när vi har fått en tår på tand, en skål.} % Insert one of the average-sized verses, used for centering the poem

\poemauthorcenter{Chalmersspexet Katarina II, år 1959} % Right-aligned author
\poemauthorcenter{\textbf{Melodi:} Kavalerijskaja} % Right-aligned author
\begin{verse}[\versewidth]

%------------------------------------------------
|: Livet är härligt,\\
tavaritj, vårt liv är härligt.\\
Vi alla våra små bekymmer glömmer,\\
när vi har fått en tår på tand, en skål.\\!

Tag dig en vodka,\\
tavaritj, en liten vodka.\\
Glasen i botten vi tillsammans tömmer.\\
Det kommer mera efter hand. :| Hej! \\!
%------------------------------------------------
\end{verse}

%----------------------------------------------------------------------------------------

\poemtitle{Be-Be vitamin}

\settowidth{\versewidth}{Mången kalori, simmar däruti.} % Insert one of the average-sized verses, used for centering the poem

\poemauthorcenter{Okänd} % Right-aligned author
\poemauthorcenter{\textbf{Melodi:} Bä bä vita lamm} % Right-aligned author
\begin{verse}[\versewidth]

%------------------------------------------------
Be-Be-vitamin, finns i brännevin\\
Mången kalori, simmar däruti.\\
Helgdagssup åt far och\\
söndagskrök åt mor samt tre små\\
huttar åt lille lille bror.\\!
%------------------------------------------------
\end{verse}

%----------------------------------------------------------------------------------------

\poemtitle{Vill ha akvaviten}

\settowidth{\versewidth}{jag borde köpa folkis men vad ska jag ta mig till} % Insert one of the average-sized verses, used for centering the poem

\poemauthorcenter{Okänd} % Right-aligned author
\poemauthorcenter{\textbf{Melodi:} Vill ha dig} % Right-aligned author
\begin{verse}[\versewidth]

%------------------------------------------------
Vi har druckit billig folköl i snart ett år \\
jag har gömt mina burkar så gott det går \\
men när du bjuder på akvavit så inser jag genast\\ 
att öl é skit\\!

Vi står i jouraffären, jag fattar ingenting \\
jag borde köpa folkis men vad ska jag ta mig till \\
när det enda som jag tänker på är len akvavit \\
en vacker sprit\\!

Å å-\\
Vill ha akvaviten hos mig \\
tiden den stannar när flaskan är tom. \\
Å jag lättar, jag flyger, jag svävar fram \\
låt den aldrig ta slut.\\!
%------------------------------------------------
\end{verse}


%----------------------------------------------------------------------------------------

\newpage

\poemtitle{Vikingen}

\settowidth{\versewidth}{Den hastigt i mitt svalg försvann, hurra, hurra!} % Insert one of the average-sized verses, used for centering the poem

\poemauthorcenter{LTH-teknologer} % Right-aligned author
\poemauthorcenter{\textbf{Melodi:}When Johnny comes marching home} % Right-aligned author
\begin{verse}[\versewidth]

%------------------------------------------------
En viking älskar livets vann, hurra, hurra!\\
Den hastigt i mitt svalg försvann, hurra, hurra! \\
Till kalv, till oxe, till fisk, till fläsk, \\
när alla kärringar vill ha läsk,\\
ja, då vill alla vikingar ha en bäsk. \\!

När bäsken småningom är slut, tragik, tragik!\\
Då bärs varenda viking ut, som lik, sig lik.\\
Och när vi vaknar, vi sjunger en bit, \\
sen korkar vi upp Skåne Akvavit.\\
Skål för alla vikingar som kom hit!\\!
%------------------------------------------------
\end{verse}

%----------------------------------------------------------------------------------------

\poemtitle{Strejk på Pripps}

\settowidth{\versewidth}{Jag drömde det var strejk på Pripps} % Insert one of the average-sized verses, used for centering the poem

\poemauthorcenter{Okänd} % Right-aligned author
\poemauthorcenter{\textbf{Melodi:}I natt jag drömde} % Right-aligned author
\begin{verse}[\versewidth]

%------------------------------------------------
I natt jag drömde något som \\
jag aldrig drömt förut.\\
Jag drömde det var strejk på Pripps\\
och alla ölen var slut.\\
Jag drömde om en jättesal där\\
ölen stod på rad.\\
Jag svepte väl ett tjugotal och\\
reste mig och sa:\\!

”Men MTS,\\
det är ju för fan min favorit\\!

%------------------------------------------------
\end{verse}

%----------------------------------------------------------------------------------------
\newpage
\poemtitle{Måsen}

\settowidth{\versewidth}{Och tungan lådde vid skepparns gom} % Insert one of the average-sized verses, used for centering the poem

\poemauthorcenter{Okänd} % Right-aligned author
\poemauthorcenter{\textbf{Melodi:}När månen vandrar} % Right-aligned author
\begin{verse}[\versewidth]

%------------------------------------------------
Det satt en mås på en klyvarbom,\\
och tom i krävan var kräket.\\
Och tungan lådde vid skepparns gom\\
där han satt uti bleket.\\
Jag vill ha sill hördes måsen rope,\\
och skepparn svarte: jag vill ha OP,\\
om blott jag får, om blott jag får.\\!

Nu lyfter måsen från klyvarbom\\
och vinden spelar i tågen.\\
OP:n svalkat har skepparns gom.\\
Jag önskar blott att jag såg ’en.\\
Så nöjd och lycklig, den arme saten,\\
han sätter storsegel den krabaten.\\
Till sjöss han far och halvan tar.\\
När månen vandrar sin tysta ban\\!

Och tittar in genom rutan\\
Då tänker jag, att på ljusan da’n,\\
då kan jag klara mig utan.\\
Då kan jag klara mig utan måne,\\
men utan Renat och utan Skåne?\\
Det vete fan, det vete fan.\\!

Den mås som satt på en klyvarbom,\\
den är nu död och begraven,\\
Och skepparn som drack en flaska rom,\\
han har nu drunknat i haven.\\
Så kan det gå om man fått för mycket,\\
om man för brännvin har fattat tycke.\\
Vi som har kvar, vi resten tar.\\!

Min kompis Anna hon är en bot \\
Hon röjer upp i kanalen \\
Och varje gång jag hör hennes låt \\
Så får jag ont i analen \\
Jag är så trött på den jävla låten \\
Kan någon vänlig själ banna boten \\
Det vette fan, jag fick en ban! \\!

Det satt en mes i en klyvarmast, \\
där sågs han ragla och svaja. \\
För trots att frön var hans enda last \\
var han nu full som en kaja. \\
''Vad har du gjort!'' hördes skepparn stöna \\
och mesen svarte: "Jag rökte fröna! \\
I egen holk, i egen holk." \\!

När nubben blänker i immigt glas \\
som hoppets strålande stjärna, \\
då är det avsett att det ska tas \\
förutan fruktan och gärna. \\
Så klang och klingom, så tar vi supen, \\
den läskar härligt den torra strupen. \\
Ja, skål gutår, ja, skål gutår! \\!

Jag tänker sälja min dromedar. \\
Jag tänker flytta till norden. \\
Vem vill va’ bosatt uti ett land, \\
där man får ligga vid borden? \\
Nu konverterar jag här på snabben! \\
Jag vill ha akvavit till kebaben! \\
Var ingen mes, fyll upp din fez! \\!

Det satt en mus i en hushållsost \\
och åt och åt utan måtta \\
tills osten blev till en mushåls-ost \\
och han en klotformad råtta. \\
''Så bra'', sa musen ''att va en fettboll \\
nu kan jag rulla med hast åt rätt håll: \\
Ostindien, Ostindien.'' \\!

Det flög en JAS över Västerbron \\
men styrsystemet var trasigt. \\
Piloten ut sköt sig med kanon \\
för planet svängde så knasigt. \\
''Jag vill ju uppåt, jag vill ju mer''\\ 
men planet svarte: ''Jag vill ju ner'' \\
''mot alla hjon på Västerbron.'' \\!


%------------------------------------------------
\end{verse}

%----------------------------------------------------------------------------------------
\restoregeometry

\end{document}